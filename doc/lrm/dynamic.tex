%%%%%%%%%%%%%%%%%%%%%%%%%%%%%%%%%%%%%%%%%%%%%%%%%%%%%%%%%%%%%%%%%%%%%%%%%%%%%%%%%%%%%%%%%
%%                                                                                     %%
%%                This file is part of the CAPH Compiler distribution                  %%
%%                            http:%/caph.univ-bpclermont.fr                           %%
%%                                                                                     %%
%%                                  Jocelyn SEROT                                      %%
%%                         Jocelyn.Serot@univ-bpclermont.fr                            %%
%%                                                                                     %%
%%         Copyright 2011-2018 Jocelyn SEROT.  All rights reserved.                    %%
%%  This file is distributed under the terms of the GNU Library General Public License %%
%%      with the special exception on linking described in file ..%LICENSE.            %%
%%                                                                                     %%
%%%%%%%%%%%%%%%%%%%%%%%%%%%%%%%%%%%%%%%%%%%%%%%%%%%%%%%%%%%%%%%%%%%%%%%%%%%%%%%%%%%%%%%%%

% \newcommand{\sy}[1]{\mathbf{#1}}
% \newcommand{\sem}[1]{\mathsf{#1}}
% \newcommand{\tm}[1]{\mathtt{#1}}
% \newcommand{\ut}[1]{\emph{#1}}
% \newcommand{\cat}[1]{\text{#1}}
% \newcommand{\semwild}{\underline{.}}

\chapter{Dynamic semantics}
\label{chap:dynamic-semantics}

This semantics is the one used to define the reference interpreter.
% Again, we present a slightly simplified one based on the \emph{Core} subset defined in
% Chap.~\ref{chap:abssyn}. Moreover, it does not include the behavior of i/o streams\footnote{Channels
% connected to input (resp. streams) are supposed to be constantly ready.}
% and does not handle the case of % uninitialized variables and
% of out-of-bound indexes in arrays.
% \footnote{
% Formally handling this situation would require introducing exceptions to the semantic
% domains; in practice, the reference interpreter actually uses exceptions to deal with this.}.

It is defined in an axiomatic style. It assumes that type-checking has
been properly carried out and that all translations defined by the static semantics are valid and
have been properly carried out.

\section{Semantic domain}

The dynamic semantics is given in terms of the semantic domain {\it DVal}
defined below. We use the same notations than for the static semantics for tuples, environments,
\emph{etc.} In addition, we denote by $E[x \mapsto y]$ the environment that maps $x$ to $y$ and behaves like $E$ otherwise. 
We write $\emptylist$ to denote the empty list and $[a,b,c]$ for the list
containing the elements $a,b \text{ and } c$. 
The concatenation of two lists is written $l_1 \listconc l_2$.
If $h$ is a suitable element and $t$ is a (possibly empty) list,
then $\prepend{h}{t}$ is the list obtained by prepending $h$ to~$t$; 
while $\append{t}{h}$ denotes the list obtained by attaching element $h$ at the end of list~$t$.
The cardinality of a list $l$ is denoted by~$|l|$.

% As in chapter~\ref{chap:static-semantics}, and when we will use the dot notation to
% access individual components of processes and channels when appropriate. 

% We will use the notation $\underline{.}$ to denote the "\emph{dont care}" (wildcard)
% value at the semantical level (in order to avoid confusion with the "\emph{dont care}" value at the
% syntactical level (\_).

% We use the classical notations for lists, writting
% $[\vv_1;\ldots;\vv_n]$ for a list composed of elements $\vv_1, \ldots \vv_n$, $\txt{hd}(l)$ for the head
% of list $l$, and $\txt{tl}(l)$ for its tail. The empty list is written $[]$. List concatenation is
% written $\l\concat l'$.

\begin{equation*}
  \renewcommand{\arraystretch}{1}
  \begin{array}[t]{c|c|l|l}
\textbf{Variable}  & \textbf{Set ranged over}  &  \textbf{Definition} & \textbf{Meaning} \\\hline 
\text{v}           & \sem{DVal}  & \sem{EVal} + \sem{Unknown} & \text{Expression-level semantic value} \\
\EE                & \sem{EEnv} & \lbrace \txt{id} \mapsto \sem{DVal} \rbrace & \text{Dynamic environments} \\
\pi                & \sem{Process} & \langle \{\sem{id}\mapsto\sem{DVal}\}, & \text{Processes} \\
                   &           &         \{\sem{id}\mapsto\sem{cid}\},~ \{\sem{id}\mapsto\sem{cid}\}, & \\
                   &           &         \{\sem{id}\mapsto\sem{DVal}\},~ \sem{Rule}^{*},~ \sem{RIndex} \rangle & \\
\text{r}           & \sem{Rule} & \ttuple{\sem{QPat}^+}{\sem{QExp}^+} & \text{Actor rule} \\
\text{qpat'}        & \sem{QPat} & \tttuple{\sem{Qual}}{\txt{id}}{\emtxt{Pat}} & \text{Qualified rule pattern}  \\
\text{qexp'}        & \sem{QExp} & \tttuple{\sem{Qual}}{\txt{id}}{\emtxt{Exp}} & \text{Qualified rule expression}  \\
\text{q}           & \sem{Qual} & \sem{In}+\sem{Out}+\sem{Var} & \text{Qualifier} \\
theta            & \sem{Channel} & \tttuple{\text{bool}}{\text{bool}}{[\sem{DVal}]} & \text{Channels} \\
\PP,\II,\AA       &              & \lbrace \sem{pid} \mapsto \sem{Process} \rbrace & \text{Process environment} \\
\HH               &              & \lbrace \sem{cid} \mapsto \sem{Channel} \rbrace & \text{Channel environment} \\
\BB               &              & \lbrace \sem{bid} \mapsto \sem{Box} \rbrace & \text{Box environment} \\
\WW               &              & \lbrace \sem{wid} \mapsto \sem{Wire} \rbrace & \text{Wire environment} \\
\text{l},\text{l'} & \sem{pid} & \lbrace 1, 2, \ldots \rbrace & \text{Process ids} \\
\text{k},\text{k'} & \sem{cid} & \lbrace 1, 2, \ldots \rbrace & \text{Channel ids} \\
\text{j}          & \sem{RIndex} & \lbrace 0, 1, \ldots \rbrace & \text{Rule index} \\
\end{array}
\end{equation*}

\medskip
\textbf{Expression-level} semantic values are the same as those defined for the static
semantics, with the addition of an $\sem{Unknown}$ value for dealing with uninitialized local
variable.

\medskip
\textbf{Processes} are the dynamic view of boxes. They consist of 
\begin{itemize}
\item a list of parameters (mapping names to values),
\item a list of inputs and outputs (mapping names to channel ids),
\item a list of local variables (mapping names to values),
\item a list of rules.
\item a rule index, indicating the current fireable rule for an active process (rules are numbered from
  $0$ to $N_r$; 0 means that no rule is fireable)
\end{itemize}

\medskip
\textbf{Channels} are the dynamic view of wires. They consist of
\begin{itemize}
\item two boolean flags indicating whether the channel is ready for reading (not empty) and
  ready for writing (not full).
\item a list of memorized values (modeling the buffering capabilities of the channel),
\end{itemize}


\section{Programs}
\label{sec:programs}

\ruleheader{\TE_0, \EE_0 \vdash \text{Program} \gives \PP,\HH}

\infrule[Program]
{ \TE_0, \EE_0 \vdash
  \sy{program}~\emtxt{valdecls}~\emtxt{actdecls}~\emtxt{strdecls}~\emtxt{netdecls}~\gives~\TE,\EE,\NE,\BB,\WW \\
  \TE,\EE,\NE \vdash \BB \gives \PP \andalso \vdash \WW \gives \HH \\
  \EE_0 \oplus \EE, \HH \vdash \PP \gives \HH', \PP' }
{\SE_0, \TE_0, \EE_0 \vdash \sy{program}~~\emtxt{valdecls}~\emtxt{actdecls}~\emtxt{strdecls}~\emtxt{netdecls}~\gives~\HH',\PP'}

The program is first translated to set of type, expression-level, network-level, box and wire
environments according to the static semantics.  Boxes and wires are turned into initial sets of
processes and channels and the processes are executed. The result is an final set of processes (in
which no process can be executed any longer) and channels.

\subsection{Conversion from boxes to processes}
\label{sec:conv-from-boxes}

\medskip
\ruleheader{\TE,\EE,\NE \vdash \sem{B} \gives \sem{P}}

\infrule[]
{\forallin{i}{1}{n},~~ \TE,\EE,\NE \vdash \emtxt{b}_i \gives \pi_i}
{\WW \vdash \mapn{l}{b} ~\gives \mapn{l}{\pi}} 

\ruleheader{\TE,\EE,\NE \vdash \sem{b} \gives \sem{\pi}}

\infrule[]
{\txt{b} =
  \tttttuple{\id}{\sem{BBox}}{\txt{params}}{\txt{vars}}{\txt{bins}}{\txt{bouts}}{\txt{rules}} \\
 \vdash \txt{bins} \gives \txt{ins} \\
 \vdash \txt{bouts} \gives \txt{outs} \\
 \text{ins},\text{outs},\text{vars} \vdash \text{rules} \gives \text{rules'}}
{\vdash \txt{l} \mapsto \txt{b} ~\gives
    \txt{l} \mapsto \tttttuple{\txt{params}}{\txt{ins}}{\txt{outs}}{\txt{vars}}{\txt{rules'}}}

\todo{Translation of $\sem{BBin}$ and $\sem{BBout}$ boxes !}

\medskip
\ruleheader{\vdash \text{bins}/\text{bouts} \gives \text{ins}/\text{outs}}

\infrule[]
{\forallin{i}{1}{n},~~ \vdash \txt{wid}_i \gives \txt{cid}_i}
{\vdash \mapn{\txt{id}}{\txt{wid}} ~\gives \mapn{\txt{id}}{\txt{cid}}}

\ruleheader{\vdash \text{wid} \gives \text{cid}}

\infrule[]
{\vdash \txt{wid}=i}
{\vdash \txt{wid} \gives i}

Identifying wire and channel ids simplifies the translation of box ios.

% \medskip
% The following rules deal with the initialisation of local variables.

% \ruleheader{\TE,\EE \vdash \text{vars} \gives \text{vars}'}

% \infrule[]
% {\forallin{i}{1}{n},~~ \TE,\EE \vdash \txt{var}_i \gives \EE_i}
% {\TE,\EE \vdash \txt{var}_1 \ldots \txt{var}_n ~\gives \C{i=1}{n}{\EE_i}}

% \ruleheader{\TE,\EE \vdash \text{var} \gives \EE'}

% \infrule[]
% {}
% {\TE,\EE \vdash \txt{id} : \ty \gives [\txt{id} \mapsto \sem{Unknown}]}

% \infrule[]
% {\TE,\EE \vdash \emtxt{exp} \gives \vv}
% {\TE,\EE \vdash \txt{id} : \ty ~=~ \emtxt{exp} \gives [\txt{id} \mapsto \vv]}

\medskip
The following rules refine the qualification of each pattern (resp. expression) appearing in the
actor rules.

\ruleheader{\txt{ins},\txt{outs},\txt{vars} \vdash \text{rules} \gives \text{Rules}}

\infrule[]
{\forallin{i}{1}{n},~~\txt{ins},\txt{outs},\txt{vars} \vdash \txt{rule}_i \gives \txt{Rule}}
{\txt{ins},\txt{outs},\txt{vars} \vdash \lbrace \text{rule}_1, \ldots, \text{rule}_n \rbrace \gives
  \lbrace \text{rule'}_1, \ldots, \text{rule'}_n \rbrace}

\ruleheader{\txt{ins},\txt{outs},\txt{vars} \vdash \txt{rule} \gives \text{Rule}}

\infrule[]
{\forallin{i}{1}{m},~~\txt{ins},\txt{outs},\txt{vars} \vdash \txt{qpat}_i \gives \txt{qpat'}_i \\
 \forallin{i}{1}{n},~~\txt{ins},\txt{outs},\txt{vars} \vdash \txt{qexp}_i \gives \txt{qexp'}_i}
{\txt{ins},\txt{outs},\txt{vars} \vdash \tuplek{\txt{qpat}}{m} \sy{\rightarrow} \tuplek{\txt{qexp}}{n}\\
 \gives \tuplek{\txt{qpat'}}{m} \sy{\rightarrow} \tuplek{\txt{qexp'}}{n}}

\ruleheader{\txt{ins},\txt{outs},\txt{vars} \vdash \txt{qpat} \gives \txt{QPat}}

\infrule[]
{\txt{ins},\txt{outs},\txt{vars} \vdash \txt{id} \gives \txt{q}}
{\txt{ins},\txt{outs},\txt{vars} \vdash \ttuple{\txt{id}}{\emtxt{pat}} \gives \tttuple{\txt{q}}{\txt{id}}{\emtxt{pat}}}

\ruleheader{\txt{ins},\txt{outs},\txt{vars} \vdash \txt{qexp} \gives \txt{QExp}}

\infrule[]
{\txt{ins},\txt{outs},\txt{vars} \vdash \txt{id} \gives \txt{q}}
{\txt{ins},\txt{outs},\txt{vars} \vdash \ttuple{\txt{id}}{\emtxt{exp}} \gives \tttuple{\txt{q}}{\txt{id}}{\emtxt{exp}}}

\ruleheader{\txt{ins},\txt{outs},\txt{vars} \vdash \txt{id} \gives \txt{q}}

\infrule[]
{ q = \begin{cases}
   \sem{In}~\text{if}~\txt{id}\in\txt{ins},\\
   \sem{Out}~\text{if}~\txt{id}\in\txt{outs},\\
   \sem{Var}~\text{if}~\txt{id}\in\txt{Dom}(\txt{vars})\\
   \end{cases}}
{\txt{ins},\txt{outs},\txt{vars} \vdash \txt{id} \gives q}

\subsection{Conversion from wires to channels}
\label{sec:conv-from-wires}

\ruleheader{\vdash \WW \gives \HH}

\infrule[]
{\forallin{i}{1}{n},~~ \vdash \emtxt{w}_i \gives \emtxt{c}_i}
{\WW \vdash \mapn{k}{w} ~\gives \mapn{k}{c}}

\ruleheader{\vdash \sem{Wire} \gives \sem{Channel}}

\infrule[]
{}
{\vdash \ttuple{\ttuple{l}{s}}{\ttuple{l'}{s'}} \gives \tttuple{\sem{false}}{\sem{true}}{[]}}

Channels are initially empty, ready for writing and herit their id from the corresponding wire.

\todo{Initial values on wires}

\section{Processes}
\label{sec:processes}

Execution of processes proceeds by successive synchronous cycles. Each execution cycle can be decomposed into
2 steps :
\begin{enumerate}
\item processes are split into two subsets : the \emph{active} (A) subset and \emph{inactive}
  (I) subset; a process is active \emph{iff} at least one of its rules is fireable
\item all active processes are executed; during execution, a process can read values from
  its input channels, updates local variables and write values to its output channels.
\end{enumerate}
Execution cycle are repeated until the active set becomes empty.  The separation of each cycle into
two steps ensures that the order in which the processes are executed at step 2 does not matter (as
long as the channel have sufficient capacity).  This synchronous style of scheduling, in which all
active processes are executed at each cycle is coherent with a ``hardware-oriented'' interpretation
of processes\footnote{A more more ``software-oriented'' view could prefer to execute only one
  process at a time.}.

\ruleheader{\EE, \HH \vdash \PP \gives \HH',\PP'}

\infrule[EvalPs]
{\EE, \HH \vdash \PP \gives \II,\AA \andalso \EE,\HH \vdash \II, \AA \gives \HH',\PP'}
{\EE, \HH \vdash \PP \gives \HH', \PP'}

\ruleheader{\EE, \HH \vdash \II, \AA \gives \HH', \PP'}

\infrule[RunPs0]
{}
{\EE, \HH \vdash \II, \{\} \gives \HH, \II}

\infrule[RunPs1]
{\AA \not= \{\} \\
 \EE \vdash \HH, \AA \gives \HH', \AA' \\
 \EE, \HH' \vdash \AA' \cup \II \gives \II', \AA'' \\
 \EE, \HH' \vdash \II', \AA'' \gives \HH'', \PP'' }
{\EE, \HH \vdash \II, \AA \gives \HH'', \PP''}

\ruleheader{\EE, \HH \vdash \PP \gives \HH', \PP'}

\infrule[ExecPs]
{\PP = \setn{\pi} \\
 \forallin{i}{1}{n},~~ \EE, \HH_{i-1} \vdash \pi_i \gives \HH_i, \pi'_i \andalso \HH_0=\HH
 \andalso \HH'=\HH_n \\
 \PP' = \setn{\pi'}}
{\EE, \HH \vdash \PP \gives \HH', \PP'}

\subsection{Identifying and marking active processes}
\label{sec:ident-active-proc}

\ruleheader{\EE, \HH \vdash \PP \gives \PP,\PP'}

\infrule[SplitPs]
{\forallin{i}{1}{n},~~ \EE, \HH \vdash \pi_i \gives \II_i,\AA_i}
{\EE, \HH \vdash \setn{\pi} \gives \U{i=1}{n}{\II_i},~ \U{i=1}{n}{\AA_i}}

\ruleheader{\EE, \HH \vdash \pi \gives \PP,\PP'}

\infrule[SplitP]
{\EE, \HH \vdash \pi \gives \beta, \pi' \andalso \sialorsinon{\beta}{\{\},\{\pi'\}}{\{\pi'\},\{\}}}
{\EE, \HH \vdash \pi = \II,~\AA}

\ruleheader{\EE, \HH \vdash \pi \gives \sem{Bool}, \pi'}

\infrule[]
{\pi=\ttttttuple{.}{.}{.}{.}{\tuplen{\txt{r}}}{.} \\
  \existsin{j}{1}{n},~~\EE, \HH, \pi \vdash \txt{r}_j \gives \sem{true} \\
  \pi' = \ttttttuple{.}{.}{.}{.}{\tuplen{\txt{r}}}{j}}
{\EE, \HH \vdash \pi \gives \sem{true}, \pi'}

\infrule[]
{\pi=\ttttttuple{.}{.}{.}{.}{\tuplen{\txt{r}}}{.} \\
  \forallin{j}{1}{n},~~\EE, \HH, \pi \vdash \txt{r}_j \gives \sem{false} \\
  \pi' = \ttttttuple{.}{.}{.}{.}{\tuplen{\txt{r}}}{0}}
{\EE, \HH \vdash \pi \gives \sem{false}, \pi'}

A process is active if at least one of its rules is fireable.
The semantics described here does not tell
  \emph{which} rule is selected when several are fireable. It is therefore non-deterministic. Making
  it deterministic is straightforward, by assigning an fixed index to each rule and by requiring
  that the rule with the lowest (or highest) index is always selected, for example. Determinism is of course a
  highly desirable property for hardware implementations. For software implementations, other
  strategies can be chosen. For example, one could implement \emph{fair} matching  -- like in
  \textsc{hume}~\cite{Hume} -- by reordering rules after selection to ensure that each fireable rule
  is eventually selected.

\ruleheader{\EE, \HH, \pi \vdash \txt{Rule} \gives \sem{Bool}}

\infrule[FireableRule]
{\forallin{i}{1}{m},~~ \HH, \pi \vdash \txt{qpat}_i \gives \beta_i \\
 \forallin{i}{1}{n},~~ \HH, \pi \vdash \txt{qexp}_i \gives \beta'_i}
{\EE, \HH, \pi \vdash \tuplen{\txt{qpat}}\rightarrow\tuplen{\txt{qexp}} \gives
  \seqopin{\beta}{\wedge}{1}{m}\wedge\seqopin{\beta'}{\wedge}{1}{n}}

A process rule is fireable if all its LHS patterns and RHS expressions are ready.

\medskip
The following rules specify when an rule pattern is ready, depending on whether this pattern is attached
to a local variable or an input channel.

\ruleheader{\HH, \pi \vdash \sem{QPat} \gives \sem{Bool}}�

\infrule[]
{\pi.vars(\txt{id})=\txt{v} \andalso \vdash_p \txt{pat}, \txt{v} \gives \beta, \EE'}
{\HH, \pi \vdash \tttuple{\sem{Var}}{\txt{id}}{\txt{pat}} \gives \beta}

Patterns attached to variables are ready as soon as the value of the variable matches the pattern.

\infrule[]
{}
{\HH, \pi \vdash \tttuple{\sem{In}}{\txt{id}}{\sy{\_}} \gives \txt{true}}

Ignored inputs are always ready.

\infrule[]
{\pi.ins(\txt{id})=\txt{k} \andalso \HH(\txt{k})=\tttuple{\txt{false}}{\semwild}{\semwild}}
{\HH, \pi \vdash \tttuple{\sem{In}}{\txt{id}}{\txt{pat}} \gives \txt{false}}

Inputs connected to empty input channels are not ready.

\infrule[]
{\txt{pat} \not= \_ \andalso \pi.ins(\txt{id})=\txt{k} \andalso \HH(\txt{k})=\tttuple{\txt{true}}{\semwild}{\prepend{v}{vs}} \\
  \vdash_p \txt{pat}, \txt{v} \gives \beta, \EE'}
{\HH, \pi \vdash \tttuple{\sem{In}}{\txt{id}}{\txt{pat}} \gives \beta}

Inputs connected to an non-empty channel are ready \emph{iff} the first available value in the
channel matches the corresponding rule pattern.

\medskip
The three following rules specify when an rule expression is ready, depending on whether this expression is attached
to a local variable or an output channel.

\ruleheader{\HH, \pi \vdash \sem{QExp} \gives \sem{Bool}}

\infrule[OutVarRdy]
{}
{\HH, \pi \vdash \tttuple{\sem{Var}}{$\semwild$}{$\semwild$} \gives \sem{true}}

Variables are always ready for updating.

\infrule[]
{}
{\HH, \pi \vdash \tttuple{\sem{Out}}{\txt{id}}{\sy{\_}} \gives \txt{true}}

Ignored outputs are always ready.

\infrule[OutChanRdy]
{\txt{pat} \not= \_ \andalso \pi.outs(\txt{id})=\txt{k} \andalso \HH(\txt{k})=\tttuple{\semwild}{\beta}{\semwild}}
{\HH, \pi \vdash \tttuple{\sem{Out}}{\txt{id}}{\txt{pat}} \gives \beta}

An non-ignored output is ready for \emph{iff} the connected channel can be written (is not full).

\subsection{Individual process execution}
\label{sec:process}

Execution of an active process is done as follows :

\begin{enumerate}
\item retrieve the index of the selected fireable rule (there must be one, since the process has been tagged as active)
\item bind the rule LHS patterns to the corresponding values (consuming the values on the involved input channels)
\item add these bindings to an environment containing global values, actor parameters and actor
  local variables,
\item evaluate the RHS of the selected rule in this environment, environment producing values to
  update outputs and local variables
\end{enumerate}

\ruleheader{\EE, \HH \vdash \pi \gives \HH', \pi'}

\infrule[ExecP]
{\pi = \tttttuple{\txt{params}}{.}{.}{\text{vars}}{\tuplen{\txt{r}}}{j} \andalso 1 \leq j \leq n \\
  \txt{r}_j=\ttuple{\txt{qpat}}{\txt{qexps}} \\
  \HH, \pi \vdash \txt{qpats} \gives \EE', \HH' \\
  \EE \oplus \EE' \oplus \txt{params} \oplus \txt{vars} \HH', \pi \vdash \txt{qexps} \gives \HH'', \pi' }
{\EE, \HH \vdash \pi \gives \HH'', \pi'}

\subsubsection{Binding of rule patterns}
\label{sec:rule-pattern-binding}

\ruleheader{\HH, \pi \vdash \sem{QPats} \gives \EE', \HH'}

\infrule[BindRPats]
{\HH_{i-1}, \pi \vdash \txt{qpat}_i \gives \EE_i, \HH_i \andalso \HH_0=\HH}
{\HH, \pi \vdash \tuplen{\txt{qpat}} \gives \C{i=1}{n}{\EE_i},~ \HH_n}

The previous rule creates a local environment by binding the patterns of the rule RHS.
The four next rules details the process of pattern matching depending on whether the pattern is
attached to a variable or an input channel.

\ruleheader{\HH, \pi \vdash \sem{QPat} \gives \EE', \HH'}

\infrule[MatchVar1]
{}
{\HH, \pi \vdash \tttuple{\sem{Var}}{\txt{id}}{\_} \gives \emptyset, \HH}

\infrule[MatchVar2]
{\txt{rpat} \not= \_ \andalso \pi.vars(\txt{id}) = \vv \andalso \vdash_p \txt{rpat}, \vv \gives \EE}
{\HH, \pi \vdash \tttuple{\sem{Var}}{\txt{id}}{\txt{rpat}} \gives \EE, \HH}

\infrule[MatchInp1]
{}
{\HH, \pi \vdash \tttuple{\sem{In}}{\txt{id}}{\_} \gives \emptyset, \HH}

\infrule[MatchInp2]
{\txt{rpat}\not=\_ \andalso \pi.ins(\txt{id}) = \txt{k} \andalso
  \HH(\txt{k})=\tttuple{\sem{true}}{\beta}{\prepend{\txt{v}}{\txt{vs}}} \\
  \txt{c}' = \tttuple{|\txt{vs}>0|}{\beta}{\txt{vs}} \\
  \vdash_p \txt{rpat}, \txt{v} \gives \txt{true}, \EE'}
{\HH, \pi \vdash \tttuple{\sem{In}}{\txt{id}}{\txt{rpat}} \gives \EE', \HH[\txt{k}\leftarrow\txt{c}']}

When a pattern is attached to an input channel, the value is consummed on this channel (except when
the pattern is \_).

\subsubsection{Pattern matching}
\label{sec:indiv-patt-match}

\medskip
Pattern matching at the rule level is handled using the following rules, where $\vdash_p \txt{rpat}, \vv
\gives \sem{Bool}, \EE, $ means that binding \emph{rpat} to value $\vv$ succeeds or not and results in a 
dynamic environment $\EE$ :

\ruleheader{\vdash_p \sem{RPat}, \txt{v} \gives \sem{Bool}, \EE}

\infrule[]
{}
{\vdash \txt{int}/\txt{bool}~ \txt{v'},~ \sem{Int}/\sem{Bool}~ \vv \gives \txt{v'}=\vv, \emptyenv}

\infrule[]
{}
{\vdash \txt{var}~\txt{v'},~ \vv \gives \txt{true}, [\txt{v'} \mapsto \vv]}

\infrule[]
{}
{\vdash \sy{\_},~ \vv \gives \txt{true}, \emptyenv}

\infrule[]
{\txt{con}\not=\txt{con}'}
{\vdash \txt{con} \tuplen{\txt{rpat}}, \txt{con}' \tuplen{\vv} \gives \txt{false}, \emptyenv}

\infrule[]
{\txt{con}=\txt{con}' \andalso \forallin{i}{1}{n},~~ \vdash_p \emtxt{rpat}_i,~ \vv_i \gives \beta_i, \EE_i}
{\vdash \txt{con} \tuplen{\txt{rpat}}, \txt{con}' \tuplen{\vv} \gives \seqopn{\beta}{\wedge},~ \C{i=1}{n}{\EE}}

\subsubsection{Evaluation of rule expressions}
\label{sec:eval-rule-expr}

\ruleheader{\EE, \HH, \pi \vdash \sem{QExps} \gives \HH', \pi'}

\infrule[EvalQExps]
{\HH_0=\HH \andalso \pi_0=\pi \andalso \forallin{i}{1}{n},~ \EE,\HH_{i-1},\pi_{i_1} \vdash \txt{qexp}_i \gives \HH_i, \pi_i}
{\EE, \HH, \pi \vdash \tuplen{\txt{qexp}} \gives \HH_n, \pi_n}

\ruleheader{\EE,\HH,\pi \vdash \sem{QExp} \gives \HH', \pi'}

\infrule[UpdVar]
{\pi = \tttuple{\ldots}{\txt{vars}}{\ldots}
  \andalso \EE \vdash \txt{exp} \gives \vv
  \andalso \pi' = \tttuple{\ldots}{\txt{vars}[\txt{id}\leftarrow\txt{v}]}{\ldots}}
{\EE,\HH,\pi \vdash \tttuple{\sem{Var}}{\txt{id}}{\txt{exp}} \gives \HH, \pi'}

A variable update is directly reflected in the process local state.

\infrule[UpdOut1]
{}
{\EE,\HH,\pi \vdash \tttuple{\sem{Out}}{\txt{id}}{\_} \gives \HH, \pi}

Nothing happens when an output is assigned the value $\_$ (ignore).

\infrule[UpdOut 2]
{\txt{exp} \not= \_
 \andalso \pi.outs(\txt{id})=\txt{k}
 \andalso \HH(\txt{k})=\tttuple{\semwild}{\sem{true}}{\txt{vs}}
  \andalso \EE \vdash \txt{exp} \gives \vv \\
 \txt{c}'=\tttuple{\sem{true}}{\sem{true}}{\append{\txt{vs}}{\txt{v}}}}
{\EE,\HH,\pi \vdash \tttuple{\sem{Out}}{\txt{id}}{\txt{exp}} \gives
  \HH[\txt{k}\leftarrow\txt{c}'], \pi}

Updating an output channel adds the value at the end of the channel buffer (FIFO behavior). The
above semantics assumes undbounded FIFOs (the written channel is always ready for
writting). Assuming FIFOs with a finite capacity $\kappa$ would just require modifying the last
premisse of the above rule into :


\begin{center}
$\txt{c}'=\tttuple{\sem{true}}{|\append{\txt{vs}}{\txt{v}}|<\kappa}{\append{\txt{vs}}{\txt{v}}}$
\end{center}


%%% Local Variables: 
%%% mode: latex
%%% TeX-master: "caph"
%%% End: 
