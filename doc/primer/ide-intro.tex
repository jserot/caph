%%%%%%%%%%%%%%%%%%%%%%%%%%%%%%%%%%%%%%%%%%%%%%%%%%%%%%%%%%%%%%%%%%%%%%%%%%%%%%%%%%%%%%%%%
%%                                                                                     %%
%%                This file is part of the CAPH Compiler distribution                  %%
%%                            http:%/caph.univ-bpclermont.fr                           %%
%%                                                                                     %%
%%                                  Jocelyn SEROT                                      %%
%%                         Jocelyn.Serot@univ-bpclermont.fr                            %%
%%                                                                                     %%
%%         Copyright 2011-2018 Jocelyn SEROT.  All rights reserved.                    %%
%%  This file is distributed under the terms of the GNU Library General Public License %%
%%      with the special exception on linking described in file ..%LICENSE.            %%
%%                                                                                     %%
%%%%%%%%%%%%%%%%%%%%%%%%%%%%%%%%%%%%%%%%%%%%%%%%%%%%%%%%%%%%%%%%%%%%%%%%%%%%%%%%%%%%%%%%%

\chapter*{Introduction}
\label{sec:ide-intro}

This part describes the \caph IDE. This IDE basically provides a Graphical user Interface (GUI) to
the \caphc compiler.

\medskip
The \caph IDE allows
\begin{itemize}
\item writing, reading and editing of \caph programs, % (\texttt{.cph}) (with syntax coloring),
\item grouping all files associated to a \caph program into \emph{projects},
\item generating and viewing graphical representations of these programs,
\item running simulations of these programs,
\item generating SystemC and VHDL code.
\end{itemize}

\medskip
\textbf{Note}. This document describes the Windows version of the \caph IDE. The IDE can also be built and used on
Unix-based systems (Linux, MacOS).

%%% Local Variables: 
%%% mode: latex
%%% TeX-master: "caph-primer"
%%% End: 
