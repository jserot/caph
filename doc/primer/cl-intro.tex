%%%%%%%%%%%%%%%%%%%%%%%%%%%%%%%%%%%%%%%%%%%%%%%%%%%%%%%%%%%%%%%%%%%%%%%%%%%%%%%%%%%%%%%%%
%%                                                                                     %%
%%                This file is part of the CAPH Compiler distribution                  %%
%%                            http:%/caph.univ-bpclermont.fr                           %%
%%                                                                                     %%
%%                                  Jocelyn SEROT                                      %%
%%                         Jocelyn.Serot@univ-bpclermont.fr                            %%
%%                                                                                     %%
%%         Copyright 2011-2018 Jocelyn SEROT.  All rights reserved.                    %%
%%  This file is distributed under the terms of the GNU Library General Public License %%
%%      with the special exception on linking described in file ..%LICENSE.            %%
%%                                                                                     %%
%%%%%%%%%%%%%%%%%%%%%%%%%%%%%%%%%%%%%%%%%%%%%%%%%%%%%%%%%%%%%%%%%%%%%%%%%%%%%%%%%%%%%%%%%

\chapter*{Introduction}
\label{sec:cl-intro}

This part describes how to use \caph in a command line based environment, using
\texttt{Makefile}s. On Unix-like platforms like Linux or MacOS, this is typically accomplished by
running the corresponding tools from within a command shell. On Windows, this can be done using Unix
emulation systems like MinGW~\cite{MinGW} or Cygwin~\cite{CygWin}.  As stated in the general
introduction, although this approach may appear a bit more complicated than the former at first
sight but it provides a way of integrating existing third-party tools, such as C++ compilers and
VHDL synthetizers, in a fully automatized design flow. 

\medskip This part assumes a basic familiarity with command line interfaces, shell programming
and \texttt{make}-based compilation flows. Aside, a knowledge in digital design (and of the VHDL
language) will help to appreciate the final products of the \caph toolset. Sections describing the
synthesis of VHDL code on FPGA requires a previous knowledge of the \textsc{altera}
Quartus~\textsc{ii} environment.

\bigskip
The following typographic conventions are followed :
\begin{itemize}
\item source code is written in gray-shaded boxes, like this :
\begin{lstlisting}[style=CaphStyle]
-- CAPH source code will appear here
\end{lstlisting}
\item \texttt{makefile}s are written in pink-shaded boxes, like this :
\begin{lstlisting}[style=MakeStyle]
-- Makefiles will appear like this
\end{lstlisting}
\item shell input (on the command line) is written like this (the character \verb|#| is the shell
  prompt) :
\begin{lstlisting}[style=BashInputStyle]
# command
\end{lstlisting}
\item shell output is written like this :
\begin{lstlisting}[style=BashOutputStyle]
shell output
\end{lstlisting}
\end{itemize}

%%% Local Variables: 
%%% mode: latex
%%% TeX-master: "caph-primer"
%%% End: 
